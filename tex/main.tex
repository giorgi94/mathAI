\documentclass{article}
\usepackage[english]{babel}
\usepackage{indentfirst}
\usepackage{tikz, pgfplots}
\usepackage{graphics}
\usepackage{amsthm}
\usepackage{amssymb}
\usepackage{amsmath}
\usepackage{verbatim}
\usepackage{mathrsfs}
\usepackage[mathcal]{euscript}
\usepackage{fancyhdr}
\usepackage{enumitem}
\usepackage{titlesec}
\usepackage{tocloft}
\usepackage[a4paper]{geometry}

\geometry{top=50pt, left=50pt, right=50pt}
\usetikzlibrary{positioning}
\usetikzlibrary{calc}


\pagestyle{fancy}
\lhead{}
\rhead{}
\chead{}
\lfoot{}
\rfoot{\thepage}
\cfoot{}

\renewcommand{\headrulewidth}{0pt}
\renewcommand{\footrulewidth}{0pt}

\renewcommand\cftsecdotsep{\cftdot}
\renewcommand\cftsubsecdotsep{\cftdot}
\renewcommand\cftsecafterpnum{\vskip5pt}
\renewcommand\cftsubsecafterpnum{\vskip5pt}
\renewcommand\cftsecfont{\normalfont}
\renewcommand\cftsecpagefont{\normalfont}
\renewcommand{\cftsecleader}{\cftdotfill{\cftsecdotsep}}



\usepackage[unicode]{hyperref}
\hypersetup{
    colorlinks=true,
    citecolor=black,
    filecolor=black,
    linkcolor=black,
    urlcolor=black,
}
\usepackage[numbered]{bookmark}
\usetikzlibrary{calc}
\titleformat{\section}[block]{\Large\thesection.~}{}{0mm}{}
\titleformat{\subsection}[block]{\it\large\thesubsection.}{}{1mm}{}
% \titleformat{\subsection}[block]{\filcenter\large}{}{1mm}{}

\titlespacing*{\section}{0mm}{10mm}{10mm}
\titlespacing*{\subsection}{0mm}{10mm}{8mm}

\def\arraystretch{1.5}
\numberwithin{equation}{section}

\newtheorem{lemma}{Lemma}[section]
\newtheorem{theorem}{Theorem}[section]
\newtheorem{hypothesis}{Hypothesis}[section]

\theoremstyle{definition}
\newtheorem{definition}{Definition}[section]

\theoremstyle{remark}
\newtheorem{example}{Example}[section]
\newtheorem{remark}{Remark}[section]


\newcommand{\dsum}{\displaystyle\sum}
\newcommand{\ceil}[1]{\left\lceil #1 \right\rceil}
\newcommand{\floor}[1]{\left\lfloor #1 \right\rfloor}
\renewcommand{\mod}[1]{~\mathrm{mod}\left(#1\right)}
\renewcommand{\emptyset}{\text{\O}}


\newcommand{\quoted}[1]{\textquotedblleft#1\textquotedblright}


\def\layersep{2.5cm}



\usepackage{color}



\begin{document}
\title{Mathematics and Artificial Neural Network}
\author{Giorgi Kakulashvili}
\date{\today}
\maketitle
\thispagestyle{fancy}

% \tableofcontents
% \newpage


\section{Linear Function Approximation}


\begin{equation}\label{eq-basic-tr}
    T(x) = f\left( Wx + b \right)
\end{equation}

\[
    W = \begin{pmatrix}
        w_{11} & w_{12} & \cdots & w_{1n} \\
        w_{21} & w_{22} & \cdots & w_{2n} \\
        \vdots & \vdots & \ddots & \vdots \\
        w_{m1} & w_{m2} & \cdots & w_{mn} \\
    \end{pmatrix},
    \quad
    b = \begin{pmatrix}
        b_{1}  \\
        b_{2} \\
        \vdots \\
        b_{m} \\
    \end{pmatrix},
    \quad
    x = \begin{pmatrix}
        x_{1}  \\
        x_{2} \\
        \vdots \\
        x_{m} \\
    \end{pmatrix}, 
    \quad
    T = \begin{pmatrix}
        t_{1}  \\
        t_{2} \\
        \vdots \\
        t_{m} \\
    \end{pmatrix}.
\]

\[
    t_{i} = f(y_{i}) = f\left(\sum_{k=1}^{m} w_{ik} x_{k} + b_{i}\right), \quad
    \frac{\partial t_{i}}{\partial w_{ik}} = x_{k} f'(y_{i}), \quad
    \frac{\partial t_{i}}{\partial b_{i}} = f'(y_{i}).
\]

Let's define loss function $L$ as

\begin{equation}\label{eq-basic-loss}
    L(\hat{t};t) = L(\hat{t}_1,\dots,\hat{t}_m;t_1,\dots,t_m) = \frac{1}{2}\sum_{i=1}^{m} \left( \hat{t}_i - t_i \right)^2.
\end{equation}

where $t$ is original vector and $\hat{t}$ is computed using \eqref{eq-basic-tr}.

\[
    \delta w_{ij} =  \frac{\partial L}{\partial w_{ij}} = 
    \sum_{i=1}^{m} \left( \hat{t}_i - t_i \right) \frac{\partial \hat{t}_{i}}{\partial w_{ij}} =
    \sum_{i=1}^{m} \left( f(y_{i}) - t_i \right)  x_{j} f'(y_{i}) ,
\]

\[
    \delta b_{i} =\frac{\partial L}{\partial b_{i}} = \sum_{i=1}^{m} \left( \hat{t}_i - t_i \right) \frac{\partial \hat{t}_{i}}{\partial b_{i}} =
     \sum_{i=1}^{m} \left( f(y_{i}) - t_i \right) f'(y_{i}).
\]

Therefore, we obtain following

\begin{align}
    w_{ij} &\leftarrow w_{ij} - \gamma \delta w_{ij}, \\
    b_{i} & \leftarrow b_{i} - \gamma \delta b_{i},
\end{align}

where $\gamma$ is some small positive constant.












\end{document}


